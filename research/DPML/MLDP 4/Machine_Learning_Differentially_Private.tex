
\documentclass[draftclsnofoot,onecolumn,twoside]{IEEEtran}
%\documentclass[conference, letterpaper]{IEEEtran}


\usepackage{ctex}
\usepackage{url}
\usepackage[pdftex,colorlinks,linkcolor=green,anchorcolor=blue,citecolor=red,backref]{hyperref}

\usepackage{graphicx}
\usepackage{subfigure}
\usepackage{caption}

\begin{document}
\hypersetup{CJKbookmarks=true}

\title{\huge{Differential Privacy and Machine Learning:\\
a Survey and Review}}

\author{\IEEEauthorblockN{Xinkui Wu}\\
\IEEEauthorblockA{\small School of Electronic Information and Communications,\\
Huazhong University of Science and Technology, Wuhan, China\\
Email: wxk96@qq.com}}


\IEEEspecialpapernotice{(\today)}

\maketitle

\section{Differentially Private Unsupervised Learning}
\subsection{K-means clustering}

    K-means clustering is widely used in many applications
such as classification and feature learning. It aims to cluster
proximate data together.

    The first differentially private K-means clustering algorithm
was proposed by Blum et al. \cite{38} as SuLQ K-means.
Observe that only two queries are required explicitly: 1) the
number of points in each new cluster and 2) the sum of the data points for each cluster
to compute the centroid.

    Privacy definitions provide ways for trading-off the privacy
of individuals in a statistical database for the utility of downstream
analysis of the data. In \cite{16}, they present Blowfish, a class of privacy definitions inspired by the Pufferfish
framework as a substantial generalization of differential privacy, that provides a rich interface for this trade-off. They
use the SuLQ K-means
mechanism with the appropriate policy specific sensitivity
and thus satisfy privacy under
the Blowfish policy while ensuring better accuracy. That's to say, using the
example of K-means clustering illustrate the gains in accuracy
for Blowfish policies having weaker sensitive information
specifications.

     In \cite{1}, they proposes an ($\epsilon$,$\delta$)-differentially private K-means clustering algorithm using
the sample and aggregate framework. The mechanism is based on the assumption
that the data are well-separated. 'Well separated' means that the
clusters can be estimated easily with a small number of samples. This is a
prerequisite of the sample and aggregate framework. The mechanism randomly
splits the training set into many subsets, runs the non-private k-means algorithm
on each subset to get many outputs, and then uses the smooth sensitivity framework
to publish the output from a dense region differentially privately. This step
preserves privacy while the underlying k-means algorithm is unchanged.

    In recent years, a large and growing body of literature has investigated
differentially private data analysis. Broadly, they can
be classified into two approaches. The interactive approach aims at
developing customized differentially private algorithms for specific
data mining tasks.  The non-interactive
approach aims at developing an approach to compute, in a differentially
private way, a synopsis of the input dataset, which can then
be used to generate a synthetic dataset, or to directly support various
data mining tasks.

    In \cite{11}��they propose to combine the
non-interactive differentially private synopsis algorithms with K-means
clustering. They introduce a hybrid approach
to differentially private data analysis, which is efficient approach
to K-means clustering that further improves these cluster centroids.

    In \cite{15}, they introduce GUPT, a platform that allows organizations
to allow external aggregate analysis on their datasets
while ensuring that data analysis is performed in a differentially
private manner. They show using results from running common
machine learning algorithms such as K-means clustering that
GUPT does not significantly affect the accuracy of data
analysis. When performing K-means clustering on the same dataset and the same iteration, the data accuracy of GUPT
is higer than the other differential privacy system(PINQ).

   Trajectory data, i.e., human mobility traces, is
extremely valuable for a wide range of mobile applications.
However, publishing raw trajectories without special sanitization
poses serious threats to individual privacy. Recently, researchers
begin to leverage differential privacy to solve this challenge.

    In \cite{5}, they aim to remove
the assumption that the trajectories contain a lot of identical prefixes or n-grams,
which is not true in many applications and propose a differentially private publishing
mechanism for more general time-series trajectories. In this paper, they use classic K-means
clustering to partition the original trajectories into m group based on their pairwise Euclidean
distances to significantly improve the efficiency. The partition approach the optimum since K-means
prefers to grouping closer trajectories. Meanwhile, adding or removing a single trajectory cannot
significantly change the partition result.

    Models of human mobility have broad applicability
in urban planning, ecology, epidemiology, and other fields. Starting
with Call Detail Records (CDRs) from a cellular telephone
network that have gone through a straightforward anonymization
procedure, the prior WHERE modeling approach produces
synthetic CDRs for a synthetic population. The accuracy of
WHERE has been validated against billions of location samples
for hundreds of thousands of cell phones in the New York and
Los Angeles metropolitan areas. In \cite{45}, they introduce DPWHERE,
which modifies WHERE by adding controlled noise
to achieve differential privacy and use differentially private
K-means clustering to cluster datas of phone users. \cite{46} is a similar paper to protect the
information of human trajectories data using differentially private K-means clustering.


    From what has been discussed above, K-means clustering algorithm is fast and simple to realize.
The time complexity is close to linear and is suitable for mining large data sets. However, in the K-means algorithm,
K is given in advance, and the selection of K value is very difficult to estimate. Most of the time, it's best not to know in advance
how many categories a given data set should fall into. Besides, in the K-means algorithm, an initial partition should be determined according
to the initial clustering center, and then the initial partition should be optimized. The selection of the initial
clustering center has a great influence on the clustering results. Once the initial value selection is poor,
the effective clustering results may not be obtained. What can be seen from the K-means algorithm framework is that
the algorithm needs to constantly adjust sample classification and calculated the adjusted new clustering center,
so when is a large amount of data, the algorithm of time overhead is very large.

\subsection{DBSCAN(Density-Based Spatial Clustering of Applications with Noise)}

    DBSCAN (Density-Based Spatial Clustering of Applications with
Noise), introduced by Ester et al., is a non-parametric, densitybased
clustering technique. Along with partitioning methods and
hierarchical clustering, DBSCAN belongs to the third category of clustering
methods and assumes that a cluster is a region in the data
space with a high density.

    One main concern for individuals participating in the data collection of personal location
history records (i.e., trajectories) is the disclosure of their location and related information when a
user queries for statistical or pattern mining results such as frequent locations derived from these
records.

    In \cite{24}, they investigate how one can achieve the privacy goal that the inclusion of his
location history in a statistical database with interesting location mining capability does not substantially
increase risk to his privacy. They use a standard spatial decomposition such
as a quadtree, with DBSCAN clustering to achieve both tree-based density-based spatial
objects clustering and to demonstrate the differential privacy mechanism on a tree-based
location mining approach. The DBSCAN density-based clustering algorithm is used to extract
the likely interesting regions.

   Another approach to apply differential privacy to location
privacy is differentially private data mining inside the traces as proposed in \cite{22}. They
introduce a way to build a quad-tree
and use it to perform a DBSCAN clustering whose result is
differentially private. Like in classical differential privacy, the
goal here is to protect the result of an aggregate function and
not to access directly to the whole data set.

    Differential privacy is a new paradigm of privacy tailored
for statistical databases independent of the adversaries' background
knowledge or computational power, which definnes
a rigorous attack model, reducing the risk of privacy disclosure
and meanwhile ensuring the availability of data
successfully.

    Based on differential privacy and data
mining techniques, many algorithms have been presented,
such as Differential Privacy Preservation K-means clustering
method (DP-Kmeans), Differential
Privacy Preservation DBSCAN clustering method
(DP-DBSCAN)\cite{23}, etc. These algorithms can achieve effective
clustering via adding noise that conforms to differential
privacy. However, when DP-Kmeans faces a dataset with
an unknown number of clusters and uneven density distribution,
the clustering effect decreases. DP-DBSCAN is more
time-consuming with less clustering for larger datasets and
smaller privacy budget parameters.

    In \cite{25}, they put forward a differential
privacy preservation multiple cores DBSCAN clustering(DP-MCDBSCAN) schema based on the powerful differential privacy
and DBSCAN algorithm for network user data to effectively leverage the privacy leakage issue in the process
of data mining, enhancing data clustering efficaciously by adding Laplace noise. Privacy analysis demonstrates
that their DP-MCDBSCAN clustering schema
can not only meet the publishers' query needs but also
prevent the data of publishers from being attacked. Different from
DP-DBSCAN, DP-MCDBSCAN solves the randomness
and blindness of DP-DBSCAN effectively by
optimizing the selection of the initial core points.


    Compared with the K-means method, DBSCAN does not need to know the number of cluster classes to be formed in advance
and can find clusters of any shape. At the same time, DBSCAN can identify noise points. DBSCAN is not sensitive to the order
 of samples in the database, that is, the input order of pattern has little effect on the result. However, for the sample
at the boundary between cluster classes, it may fluctuate according to which cluster class is detected first and whose attribution.
Besides��DBSCAN does not reflect high dimensional data well and if the density of the sample set is not uniform and the cluster spacing
 difference is large, the cluster quality is poor.


\subsection{Hierarchical Clustering}

    In order to avoid the problem of K value selection and initial clustering center selection in K-means clustering method,
the hierarchical clustering is a kind of clustering that can be divided into large clusters from top to bottom which is
called division method. It can also aggregate small categories from bottom to top, which is called condensation method.
However, the most commonly used method is the condensation method from the bottom up.

    In \cite{39}��it mentions that cluster analysis is the process of grouping a set of data
objects into multiple groups or clusters so that objects within
a cluster have high similarity, but are very dissimilar to
objects in other clusters. Dissimilarities and similarities are
assessed based on the attribute values describing the objects
and often involve distance measures. Similar to the K-means clustering method, the hierarchical clustering method also
can be applied to differential privacy data protection and it doesn't need select k value and initial clustering center.

    The integration of information dispersed among multiple
repositories is a crucial step for accurate data analysis in
various domains. In support of this goal, it is critical to
devise procedures for identifying similar records across distinct data sources.
In \cite{26}, they introduce a novel model for practical private record linkage
(PRL), which 1) affords controlled and limited information
leakage, 2) avoids false matches resulting from data transformation.
Besides, applied obfuscation relies on differential privacy which provides
strong privacy guarantees against adversaries with arbitrary
background knowledge. Once the partitioning identifier is selected, the third party Charlie who helps the execution of the protocol without
learning any record from either A or B applies agglomerative hierarchical clustering on the set of values in
the public domain of the partitioning identifier. Partitioning the data sources into blocks
is to eliminate comparisons for records that are unlikely to
match. During blocking, Charlie forms the
global block basis through a hierarchical clustering over the
domain of the selected public identifier. The hierarchical clustering can make blocking performed efficiently
without significant computational cost.

    Database as a service (DAS) model where
data management is outsourced to cloud service providers
has become more prevalent. Although DAS model offers
lower cost and flexibility, it necessitates the transfer of po-
tentially sensitive data to untrusted cloud servers.

    In \cite{27}, they introduce a novel construction for practical
 and privacy-aware selective record retrieval over encrypted databases.
Their approach leaks obfuscated access
pattern to enable efficient retrieval while ensuring individual privacy.
Applied obfuscation is based on differential privacy which provides rigorous
individual privacy guarantees against adversaries with arbitrary background knowledge.
To reduce overall obfuscation cost by distributing initial
query interfaces among distinct replicas is an instance of a
clustering problem. Their replication strategy is based
on a common clustering technique, known as agglomerative
hierarchical clustering. Initially, a distinct cluster is
formed for each query interface.After
initial construction, clusters are successively combined until
no improvement is achieved on the obfuscation cost. Once
final query interfaces and associated replicas are identified,
private indexes are formed on them.

    Hierarchical clustering is often portrayed as the better
quality clustering approach, but is limited because of its quadratic
time complexity. In contrast, K-means and its variants have a
time complexity that is linear in the number of documents, but are
thought to produce inferior clusters. That's to say, K-means is used because of
its run-time efficiency and agglomerative hierarchical clustering is
used because of its quality.

\section{Differentially Private Dimensionality Reduction}
\subsection{Feature Selection}

    So-called dimension disaster is when the feature dimension exceeds a certain limit,
the performance of the classifier with the increase of feature dimension is worse instead.
The higher the dimension, the greater the time cost of training model.
What leads to the decrease the perfomance of the classifier is often because the cause of the high dimension features
contain irrelevant and redundant features, so the main purpose of the feature selection is to remove
features of irrelevant and redundant features.

    Irrelevant feature refers to the feature that is irrelevant to the current learning task
(the information provided by this feature is useless for the current learning task).
For the student achievement, the student number is irrelevant feature. Redundant feature refers to
the characteristics of the information contained in can marked out from other aspects, such as for the
characteristics of "area", can from "long" and "wide", then it is redundant feature.

    In \cite{21}, they propose an $\epsilon$-differentially private feature selection, PrivateKD, for classification.
PrivateKD is based on the assumption that all features are categorical
and each feature has finite possible values. For any set of features $S$, it defines
a function $F(S)$ which tells how many pairs of samples from different classes
can features in $S$ distinguish. The set of selected features $S'$ is initialized to $\Phi$.
Then a greedy algorithm adds new features one by one to $S'$. When selecting
a feature to add, the algorithm uses the exponential mechanism to select the
feature that can lead to the largest increase of $F(S')$. The paper provides a
utility guarantee for the special case where the cardinality of sample space $m$
and the number of features $d$ have the relation $m = d - 1$. In that case, except
probability $O(1/poly(m))$ ($poly(m)$ means a polynomial expression of $m$),
$F(S')$ �� $(1 - 1/e)*F(Soptimal)-$ $O(logm/$$\epsilon$$)$ .

    The privacy-preserving data analysis has been gained significant interest across several research
communities. The current researches mainly focus on privacy-preserving classification and regression.
However, feature selection is also an essential component for data analysis,
which can be used to reduce the data dimensionality and can be utilized to discover knowledge,
such as inherent variables in data. In \cite{40}, in order to efficiently mine sensitive data,
a privacy preserving feature selection algorithm is proposed and analyzed in theory based on local learning
and differential privacy.

    Within exiting privacy models, differential privacy is considered one of the
strongest privacy protection techniques that does not make any assumption about the attacker's background knowledge.
This approach, however, is not suitable for
high-dimensional data with large domains as the added noise substantially destroys the utility of the data.

    There are two broad categories of feature
selection techniques, namely, filters and wrappers. Filter approach attempts to assess the merits of features from the
data without considering the induction algorithm. The wrapper model, on the other hand, uses a target learning
algorithm in order to estimate the worth of attribute subsets.

    In \cite{41}, they propose the TOP-Diff algorithm which offers a trade-off between anonymization level K and the privacy budget $\epsilon$,
and enables us to publish privacy preserving datasets with high utility. They propose a novel technique for privacy preserving data publishing satisfying differential privacy
and use feature selection in order to minimize the negative impact of injecting noise into the contingency table.  They propose the TOP-Diff algorithm which incorporates
the ultimate usage of the data and employs feature selection in order to achieve a differentially private dataset on
K-anonymous dataset, thus maintaining high utility for further data
analysis according to a given task.

    In \cite{47}, they
focus on filter methods so that an analyst does not need to know
the internals of the private classifier. Filters compute a ranking or a
score for features based on their correlation with their label. Filter
methods may rank/score individual features or sets of features. They
focus on methods that score individual features. Features can be
selected by choosing the top values or those above some threshold. They show that
private feature selection indeed leads to significant improvement
in the classifiers prediction accuracy with two differentially
private classifiers (Naive Bayes and logistic regression)

    In \cite{37}, they propose an ($\epsilon$, $\delta$)-differentially private algorithm for feature selection
when the target function is stable. Unlike the previous paper, this paper doesn't
explicitly state the algorithm for feature selection. Instead, it only requires the
selection algorithm to be stable. By 'stable', it means that either the value of
function as calculated on the input dataset doesn't change when some samples
in the set change, or that the function can output the same result on a random
subset from the input dataset with high probability.

     In conclusion, feature selection brings the immediate effects of speeding up a machine learning or data mining algorithm,
improving learning accuracy, and enhancing model comprehensibility. Various studies show that features can be
removed without performance deterioration. Moreover, feature selection also leads to better data visualization,
reduction of measurement and storage requirements.

\subsection{Principal Component Analysis}

    Principal Component Analysis (PCA) is a popular method in dimension reduction.
It finds $k$ orthogonal directions on which the projections of data have
largest variance. The original data can then be represented by its projection
onto those $k$ directions. Usually $k$ is much smaller than the dimension of sample
space. Thus the projection greatly reduces the dimensionality of the data.
It is well-known that this analysis is closely related to eigen-decomposition: if
we rank the eigenvectors of matrix $A = Var[X]$ according to the corresponding
eigenvalues $\lambda_{1}$ $\ge$ $\lambda_{2}$ $\ge$ $...$ $\ge$ $\lambda_{p}$, 
then the first $k$ eigenvectors are the $k$ directions.
    
     The extant
privacy-preserving PCA algorithms have been devised based
on two major features: the notion of differential privacy
and the stage of randomization.

    The notion of differential privacy has two types: ($\epsilon$, 0)-DP
(also called pure DP) and ($\epsilon$, $\delta$)-DP (also called approximate
DP). ($\epsilon$, $\delta$)-DP is a weaker version of ($\epsilon$, 0)-DP as the former
allows the privacy guarantee to be broken with tiny probability
(more precisely, $\delta$). In the seminal work on privacypreserving
PCA (Dwork et al. 2014; Hardt and Roth 2012;
2013; Hardt and Price 2014; Blum et al. 2005), the authors
used the notion of ($\epsilon$, $\delta$)-DP. In contrast, there is
only a few work (Chaudhuri, Sarwate, and Sinha 2012;
Kapralov and Talwar 2013), which is based on ($\epsilon$, 0)-DP.

    In terms of the stage of randomization, there are two
mainstream classes of approaches. The first is randomly
computing the eigenspace (Hardt and Roth 2013; Hardt and
Price 2014; Chaudhuri, Sarwate, and Sinha 2012; Kapralov
and Talwar 2013). The noise is added in the computing procedure. 
An alternative way is directly adding noise to the covariance
matrix. Then one runs the non-private eigenspace
computing algorithm to produce the output. This class of
approaches is called input perturbation. The input perturbation algorithms publish
a noisy sample covariance matrix before computing the
eigenspace. Thus, any further operation on the noisy covariance
matrix does not violate privacy guarantee. So far as
the flexibility is concerned, the input perturbation has better
performance because it is not limited only to computing
eigenspace. Besides, the input perturbation approach is efficient
because it merely takes extra efforts on generating
the noise.
    
    In \cite{18}, they propose a new input perturbation mechanism
for publishing a covariance matrix to achieve ($\epsilon$, 0)-
differential privacy. Their mechanism uses a Wishart distribution
to generate matrix noise. In particular, they
apply this mechanism to principal component analysis.
Besides, they show that the Wishart mechanism is effective input perturbation
approach for ($\epsilon$, 0)-differentially private PCA. PCA reduces the data dimension
while keeping the optimal variance. More specifically,
it finds a projection matrix by computing a low rank approximation
to the sample covariance matrix of the given data
points. In the scenario of PCA,
their Wishart mechanism adds less noise than the Laplace,
which leads to better utility guarantee. Compared with the
privacy-preserving PCA algorithm in (Chaudhuri, Sarwate,
and Sinha 2012), their mechanism has reliable rank-$k$ utility
guarantee while the former (Chaudhuri, Sarwate, and
Sinha 2012) only has rank-1.
    
    There are two differentially private mechanisms to select eigenvectors iteratively.
The iterative methods are based on the spectral decomposition, which
ensures that if the components corresponding to the first $i - 1$ eigenvectors of
$A$ are subtracted from $A$, then the $i$-th largest eigenvector becomes the largest
eigenvector of what remains. Therefore the process of selecting the largest $k$
eigenvectors can be replaced by repeatedly finding the first eigenvector and removing
the component corresponding to the selected eigenvector. The following
two mechanisms both make use of this idea but differ on how to select the first
eigenvector.

    An ($\epsilon$, $\delta$)-differentially private mechanism is proposed in \cite{9}. The mechanism
uses the power method: $A^{n}v$ $/$ $||A^{n}v||$ converges to the first eigenvector of
$A$ if $v$ is not orthogonal to the first eigenvector. It randomly starts with a unitlength
vector $v$, then iteratively updates $v$ with $(A_{v} + $$\eta_{i}$$)$ $/$ $||(A_{v} + $$\eta_{i}$$)||$ while
$\eta_{i}$ is Gaussian noise in the $i$-th iteration. Since it is exceedingly improbable
that a random vector is orthogonal to the first eigenvector, the vector $v$ will
get close to the first eigenvector. However due to the noise, $v$ cannot converge
with arbitrary accuracy. Thus it outputs $v$ after a fixed number of iterations
and proceed to find the next largest eigenvector. A utility guarantee is provided
on the power method, which outputs the first eigenvector. However there is
no direct guarantee on the $k$ eigenvectors. For each eigenvector a output by
running the power method on matrix $A$, the distance from the first eigenvector
($||A_{a}||$ $/$ $||a||$ $-$ $\lambda_{1}$ while $\lambda_{1}$ is the first eigenvalue) 
is $O$$\left ( \left (  \sqrt{log\left ( 1/\delta  \right )}log n\right )/\epsilon  \right )$

    \cite{10} provides an $\epsilon$-differentially private mechanism for principal component
analysis. According to the property that the first eigenvector $v$ of $A$ is the unitlength
vector that maximizes $v^{T}Av$, the mechanism uses $H(X, v)$ $=$  $v^{T}Av$ as
the score function in the exponential mechanism to select the first eigenvector
from the set $\left \{ v: v^{T}v=1\right \}$ differentially privately. The selection algorithm is
specially designed to be computable in reasonable time. This paper also provides
two proofs on utility of this mechanism. For any  $0 \textless \delta \textless 1$ and privacy budget
$\epsilon$, if the first eigenvalue of matrix $A$, $\lambda_{1}> O\left (ln\left ( 1 /\delta  \right )/\left ( n\epsilon\delta   \right )  \right )$, 
then the first
eigenvector $v$ has the property $E\left [  v^{T}Av\right ]\geq \left ( 1-\delta   \right )\lambda_{1}$. 
For any $0 \textless \delta \textless 1$ and
privacy budget $\epsilon$, if the first eigenvalue 
$\lambda_{1}>O\left (  1/\left ( n\epsilon \delta ^{6} \right )\right )$, 
the $k+1$-th eigenvalue
is denoted $\lambda_{k+1}$, and the $k$-rank approximation matrix output is denoted $A_{k}$,
then the largest eigenvalue of $A-A_{k}$ is smaller than $\lambda_{k+1}+\delta \lambda_{1}$ with large
probability.
    
    Often, the training of models requires large, representative
datasets, which may be crowdsourced and contain sensitive
information. The models should not expose private information
in these datasets. In \cite{2}, they develop new
algorithmic techniques for learning and a refined analysis of
privacy costs within the framework of differential privacy.
PCA is a useful method for capturing the main features
of the input data. They implement the differentially private
PCA algorithm. They incur a privacy cost due to running a PCA. However,
they find it useful for both improving the model quality and for
reducing the training time, as suggested by their experiments
on the MNIST data. Accuracy is quite stable
over a large range of choices for the projection dimensions
and the noise level used in the PCA stage.

    In \cite{8}, they investigate the theory and
empirical performance of differentially private approximations to PCA and propose a new method
which explicitly optimizes the utility of the output. The PCA algorithm is very sensitive in this sense 
because the top eigenvector can change
by 90 degree by changing one point in the data set. They assume
that the algorithm is given $n$ data points and a target dimension $k$ and must produce a $k$-dimensional
subspace that approximates that produced by the standard PCA problem. They propose a new algorithm,
PPCA, which is an instance of the exponential mechanism of McSherry and Talwar. PPCA explicitly 
takes into account the quality of approximation that outputs a $k$-dimensional
subspace which is biased towards subspaces close to the output of PCA.
By combining results on the exponential mechanism along 
with properties of PCA algorithm,
they can show that this procedure is differentially private. They propose the algorithm,
PPCA, which is an instance of the exponential mechanism and show
that PPCA is nearly optimal.

    In \cite{17}, they consider the problem of privately releasing a low dimensional
approximation to a set of data records, represented as a matrix $A$
in which each row corresponds to an individual and each column
to an attribute. Their goal is to compute a subspace that captures the
covariance of $A$ as much as possible, classically known as principal
component analysis (PCA). They assume that each row of $A$ has
$L_{2}$ norm bounded by one, and the privacy guarantee is defined with
respect to addition or removal of any single row. They show that the
well-known, but misnamed, randomized response algorithm, with
properly tuned parameters, provides nearly optimal additive quality
gap compared to the best possible singular subspace of $A$. They
further show that when $A^{T}A$ has a large eigenvalue gap �C a reason
often cited for PCA - the quality improves significantly. In practice, allowing such "overpowering" individuals
often goes against the purpose of PCA for discovering the global
structure of many data records, and row normalization is often recommended
before applying PCA.

    Principal component analysis is the simplest method to analyze multivariate statistical distribution by eigenvalue.
The result can be interpreted as an explanation of the variance in the original data, which
direction of data value has the greatest influence on the other side difference. In other words,
PCA provides an effective way to reduce data dimensions. If the analyst removes the components corresponding
to the smallest eigenvalues from the original data, the resulting low-dimensional data must be optimized.
    
    Except principal component analysis (PCA), 
linear discriminant analysis (LDA) is widely used as the representative feature projection technique.
PCA is for unsupervised learning and LDA is for classification.
PCA is an orthonormal transformation of data to a low-dimensional space such that
maximum variance of the original data is preserved. PCA works well for
a lot of data analysis problems but does not fit well for classification purposes.
LDA and other supervised feature selection techniques are better positioned for classification in
that they reduce the input features in such a way that maximum separability between target classes is preserved.

    A reasonable compromise of privacy and utility exists at an "appropriate" resolution of the data.
In \cite{32}, they proposed novel mechanisms to achieve privacy preserving data publishing (PPDP) satisfying 
$\epsilon$-differential privacy with improved utility through component analysis. The mechanisms studied in 
this article are Principal
Component Analysis (PCA) and Linear Discriminant Analysis (LDA). The differential PCA-based PPDP
serves as a general-purpose data dissemination tool that guarantees better utility (i.e., smaller error) compared to
Laplacian and Exponential mechanisms using the same "privacy budget". Their second mechanism, the differential
LDA-based PPDP, favors data dissemination for classification purposes.

\subsection{Singular Value Decomposition}

    PCA is the solution to the eigenvector of the covariance matrix,
which happens to be the representation of the covariance.
The eigenvector solution is the process of Singular Value Decomposition(SVD), and
the column vectors of the decomposed matrix correspond to the eigenvectors.
So the solution of the covariance of PCA becomes the sample matrix of SVD.

    In the setting of large data analysis, one of the desired goals is to design a sub-linear space algorithm
to perform computation while receiving the data online. There are many non-private algorithms in this
setting. Such algorithms are called online algorithms or streaming algorithms.
    
    The focus of \cite{33} is on differential privacy of streaming data using sketch-based algorithms. They
work in the well studied streaming model of computation, where
the database is stored in the form of a matrix and a curator can access the database row-wise or
column-wise.  It is common to perform a procedure that computes
a full or partial singular value decomposition (SVD) of the covariance matrix corresponding to the input
data matrix, and then appeal to standard statistical model selection criterion, like getting the top half of
the spectrum of the matrix, to quantify its significance.

    \cite{36} proves that an "old dog", namely, the classical Johnson-Lindenstrauss transform,
"performs new tricks", it gives a novel way of preserving differential privacy. They show that
if taking two databases, $D$ and $D'$ , such that 1) $D-D'$   is a rank-1 matrix of bounded norm
and 2) all singular values of $D$ and $D'$ are sufficiently large, then multiplying either $D$ or $D'$
with a vector of iid normal Gaussians yields two statistically close distributions in the sense of
differential privacy. Furthermore, a small, deterministic and public alteration of the input is
enough to assert that all singular values of $D$ are large. And SVD is used to solve transformation of
matrix in order to decrease the dimensionality. Besides, throughout those papers\cite{42}\cite{43}\cite{44}, 
they use the singular value decomposition
of a matrix, which is a classic tool of matrix analysis. 
                             
    Low rank approximation is a fundamental computational
primitive widely used in data analysis. In many
applications the dataset that the algorithm operates on
may contain sensitive information about contributing
individuals (e.g. user/movie ratings in the Net
ix challenge),
motivating the need to design low rank approximation
algorithms that preserve privacy of individual
entries of the input matrix. 
    
    In \cite{42}, they give a polynomial time algorithm
that, given a privacy parameter $\epsilon > 0$, for a symmetric
matrix $A$, outputs an $\epsilon$-differentially approximation to
the principal eigenvector of $A$, and then show how this
algorithm can be used to obtain a differentially private
rank-$k$ approximation. They give algorithms for releasing an $\epsilon$-differentially 
private low-rank approximation to a symmetric
matrix. Their algorithm for general $k$ relies on
an algorithm for outputting a differentially private vector
most of whose mass is concentrated in the few most
signifficant eigenvalues. They then use this primitive to iteratively
subtract rank-1 approximations from the input
matrix, obtaining an approximate differentially private
SVD. 


\bibliographystyle{IEEEtran}
\bibliography{mybib}


\end{document}


